
\section{Introduction}
\label{sec:introduction}

The quantitative understanding of the physical world is an essential goal of geoscience research. We use mathematical abstractions to represent the behavior of systems under static and dynamic conditions; and we define properties such as density and elastic moduli to characterize the capacity of materials to absorb or transmit forces in static and dynamic processes. In seismology and geophysics, our understanding of physical phenomena associated with earthquakes, their genesis, and effects, depends, in good measure, on our knowledge and accurate representation of the geometry and material properties of the Earth's structure, as well as on our capacity to represent the mechanical characteristics of the earthquake rupture process and the subsequent propagation of seismic waves through the earth. Computational geoscientists use stress conditions and dynamic rupture models to describe faulting processes, and seismic velocity and attenuation models, along with wave propagation principles, to calculate characteristics of earthquake ground motions. Initial stress models and seismic velocity models are therefore basic inputs used in earthquake ground motion simulation.

We are interested in how seismic velocity models are built and made available to geoscientists, and how these models can help advance physics-based earthquake science. We utilize modeling approaches based on deterministic numerical techniques---such as the finite element, finite difference, or spectral element methods---to simulate the ground motion in ways that incorporate the physics of earthquake processes explicitly, that is, methods that explicitly solve the associated wave propagation problem. The use of physics-based earthquake simulation has increased considerably over the last two decades thanks to the growth in capacity and availability of high-performance computing (HPC) facilities and applications \citep[e.g.,][]{Aagaard_2008_BSSA2, Olsen_2009_GRL, Bielak_2010_GJI, Cui_2010_Proc}. These simulations have important applications in seismology and earthquake engineering for purposes such as the assessment of regional seismic hazard \citep[e.g.,][]{Graves_2011_PAG}.

Recent earthquake simulations have shown the importance of velocity models in the accuracy of simulation results \citep[e.g.,][]{Taborda_2014_BSSA}. Numerous seismic velocity models have been built for specific regional or local structures and used in particular simulations over the years \citep[e.g.,][]{Frankel_1992_BSSA, Brocher_2008_BSSA, Graves_2008_BSSA}. The concept of community velocity models (CVMs) has emerged from broad use of velocity models in earthquake simulations. CVMs are seismic velocity models that have been developed, maintained, improved, and used by a community of interested investigators. For example, CVMs have been created for regions of southern and northern California \citep[][]{Kohler_2003_BSSA, Suss_2003_JGR, Brocher_2006_Proc, Shaw_2015_EPSL}, Utah \citep[][]{Magistrale_2006_Tech}, and the central United States \citep[][]{RamirezGuzman_2012_BSSA}.

CVMs are typically distributed in the form of datasets or collections of files, or in the form of computer programs that can dynamically operate on these datasets and files to provide information about the geometry and material properties of the crust in a particular region. However, these datasets and computer programs have not been designed consistently from a computational perspective. For example, not all CVMs define the same material properties, or use the same geographical projection. In addition, recent advances in earthquake simulations, powered by the increasing capability of supercomputers, have increased significantly the computational demand placed on CVMs as input to these simulations.

This paper reviews the Unified Community Velocity Model (UCVM), a software framework developed and maintained by the Southern California Earthquake Center (SCEC), designed to provide standardized and computationally efficient access to seismic velocity models. UCVM is a collection of software tools and application programming interfaces (APIs) that facilitate access to the material properties stored in CVMs. Although UCVM was conceived as a tool to aid physics-based earthquake ground-motion simulation and regional seismic hazard assessment, it can be, and has been used in other geoscience and engineering applications. Here, we review the development of UCVM and its various software components and features, including its use in high-performance parallel computers, and illustrate its capabilities with examples of recent applications of UCVM tools in geoscience and earthquake engineering research.

