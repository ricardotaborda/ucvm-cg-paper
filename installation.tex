
\section{Download and Installation}
\label{sec:installation}

The latest version of the UCVM platform is available at:
%
\url{http://hypocenter.usc.edu/research/ucvm/14.3.0/ucvm-14.3.0.tar.gz}. 
\textcolor{red}{We must use a non-versioned URL that always points to the latest version.}
%
After successfully downloading this package, the user can follow one of two possible options: the \textit{easy method} or the \textit{custom method}. The easy method uses a Python wrapper to facilitate the installation process that prompts the user about installation paths and default models to be enabled. Figure \ref{fig:instaeasy} shows the sequence of terminal commands used during the easy installation process in a Linux workstation.


\begin{figure}[th]
\begin{lstlisting}[
	frame=single,
	basewidth={0.45em,0.4em},
	backgroundcolor=\color{mylistingbkgd},
	basicstyle=\ttfamily\footnotesize,breaklines=true,
	linewidth=0.98\columnwidth,xleftmargin=0.07\columnwidth,
	numbers=left,numberblanklines=true,numberstyle=\scriptsize\color{mylistingnclr}]
> tar zxvf ucvm-14.3.0.tar.gz
> cd ./UCVM
> ./ucvm_setup.py
  It looks like you are installing UCVM for the first time.
  Where would you like UCVM to be installed?
  Enter path or blank to use default: 

  <user path entry>

  Would you like to download and install CVM-S4?
  Enter yes or no: 

  <yes,no>

  [...]
\end{lstlisting}
\caption{Easy installation procedure. Shell commands are indicated by the symbol \textgreater. \textcolor{red}{This example needs to be completed with a full installation sequence.}}
\label{fig:instaeasy}
\end{figure}

Alternatively, one can also customize the installation by directly invoking the underlying Automake \emph{configure} script. An example situation is the case when a user wishes to install a version of a CVM other than the latest supported (regression to a prior CVM release). In this case, the user must pre-install the CVMs and library dependencies themselves. When installing UCVM manually, three flags need to be provided for each model to be properly configured, namely the paths to the model data files, API library, and API include files. Similarly, the Proj4 and Euclide etree libraries must be specified with the paths to their library and include files. 

\textcolor{green}{Refer reader to the user guide as this outside scope.}

\textcolor{green}{Commented out last paragraph and fig 3 as it is too detailed.}
%As an example, Figure \ref{fig:instadvanced} shows the sequence of commands an advanced user would need to manually install UCVM in a Cray supercomputer system with default PGI compilers with static build and IO buffering, enabling the Proj4 and Etree libraries and the model CVM-S. Note that UCVM requires GNU GCC compiler 4.3+; therefore, in the example in Figure \ref{fig:instadvanced}, the user needs to switch to the GNU compilers. When installing UCVM manually, three flags need to be provided for each model to be properly configured, namely a \texttt{lib-path}, a \texttt{model-path}, and an \texttt{include-path}. Similarly, each library needs a \texttt{lib-path} and an \texttt{include-path}. The last command in the example in Figure \ref{fig:instadvanced}, \texttt{make check} will test the installation to make sure that UCVM was deployed correctly. 


%
\begin{figure}[ht!]
\centering
\begin{lstlisting}[
	frame=single,
	basewidth={0.45em,0.4em},
	backgroundcolor=\color{mylistingbkgd},
	basicstyle=\ttfamily\footnotesize,breaklines=true,
	linewidth=0.98\columnwidth,xleftmargin=0.07\columnwidth,
	numbers=left,numberblanklines=true,numberstyle=\scriptsize\color{mylistingnclr}]
> module swap PrgEnv-pgi PrgEnv-gnu
> module load iobuf
  [...]
> UCVM_DIR=<user defined path>
> UCVM_VERSION=14.3.0
> ROOT_URL=http://hypocenter.usc.edu/research/ucvm
> LIBS_URL=$ROOT_URL/$UCVM_VERSION/libraries
> MODELS_URL=$ROOT_URL/$UCVM_VERSION/models
  [...]
> cd $UCVM_DIR
> wget $LIBS_URL/proj-4.8.0.tar.gz
> wget $LIBS_URL/euclid3-1.3.tar.gz
> wget $MODELS_URL/cvms4.tar.gz
  [...]
> tar xzvf proj-4.8.0.tar.gz
> cd proj-4.8.0
> ./configure --prefix=$UCVM_DIR/lib/proj-4 --with-jni=no
> make
> make install
  [...]
> cd $UCVM_DIR
> tar xvzf euclid3-1.3.tar.gz
> cd euclid3-1.3
> ./configure --prefix=$UCVM_DIR/lib/euclid3
> make; make install
  [...]
> cd $UCVM_DIR
> tar xzvf cvms4.tar.gz
> cd CVM-S
> ./configure --prefix=$UCVM_DIR/model/cvms4
> make
> make install
  [...]
> cd $UCVM_DIR
> tar xzvf ucvm-14.3.0.tar.gz
> cd UCVM
> ./configure \
  --prefix=$UCVM_DIR \
  --enable-iobuf \
  --enable-static \
  --with-etree-include-path=$UCVM_DIR/lib/euclid3/include \
  --with-etree-lib-path=$UCVM_DIR/lib/euclid3/lib \
  --with-proj4-include-path=$UCVM_DIR/lib/proj4/include \
  --with-proj4-lib-path=$UCVM_DIR/lib/proj4/lib \
  --enable-model-cvms \
  --with-cvms-lib-path=$UCVM_DIR/model/cvms4/lib \
  --with-cvms-model-path=$UCVM_DIR/model/cvms4/src \
  --with-cvms-include-path=$UCVM_DIR/model/cvms4/src
> make
> make install
> make check
  [...]
\end{lstlisting}
\caption{Advanced installation procedure showing the commands a user will follow when working on a Cray supercomputer system with default PGI compilers and Lustre filesystem. Note that UCVM requires GNU GCC compilers, version 4.3+. This particular set of instructions installs the model CVM-S (version 4) with the Proj and Etree libraries using static build and enabling IO-buffering. Shell commands are indicated by the symbol \textgreater. \textcolor{red}{This example needs to be checked.}}
\label{fig:instadvanced}
\end{figure}




