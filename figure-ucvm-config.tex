
\begin{figure}[th]
\begin{lstlisting}[
	frame=single,
	basewidth={0.45em,0.4em},
	backgroundcolor=\color{mylistingbkgd},
	basicstyle=\ttfamily\footnotesize,breaklines=true,
	linewidth=0.98\columnwidth,xleftmargin=0.07\columnwidth,
	numbers=left,numberblanklines=true,numberstyle=\scriptsize\color{mylistingnclr}]
# UCVM config file

# UCVM model path
ucvm_interface=map_etree
ucvm_mappath=/path/to/ucvm-14.3.0/model/ucvm/ucvm.e

# SCEC CVM-S
cvms_modelpath=/path/to/ucvm-14.3.0/model/cvms4/src

# SCEC CVM-H
cvmh_modelpath=/path/to/ucvm-14.3.0/model/cvmh1191/model

# 1D
1d_modelpath=/path/to/ucvm-14.3.0/model/1d/1d.conf

# 1D GTL
1dgtl_modelpath=/path/to/ucvm-14.3.0/model/1d/1d.conf

# Model flags
cvmh_param=USE_1D_BKG,True
cvmh_param=USE_GTL,True
\end{lstlisting}
\caption{UCVM configuration file example. Before running UCVM commands, the user needs to prepare a configuration file. This file is used by UCVM to identify what models and libraries are being used by the client. This example corresponds to a UCVM installation where only the models CVM-S and CVM-H were enabled.}
\label{fig:ucvm-config}
\end{figure}
