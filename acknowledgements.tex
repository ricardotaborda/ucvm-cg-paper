
\section{Acknowledgements}

We want to extend a special acknowledgement to community velocity modelers Harold Magistrale, John Shaw, Andreas Plesch, Peter S\"{u}ss, Carl Tape, Guoqing Lin, En-Jui Lee, Po Chen, Robert W.~Graves, and many other model contributors. UCVM would not be possible without your models and your input at various development stages during the years has been invaluable. We also want to thank the Carnegie Mellon etree library development group led by David O'Hallaron for their multiple direct and indirect contributions to the development of UCVM.

This work was supported by National Science Foundation (NSF) awards: ``Geoinformatics: A Petascale Cyberfacility for Physics-Based Seismic Hazard Analysis (SCEC PetaSHA3Project)'' (EAR-0949443); ``SI2-SSI: A Sustainable Community Software Framework for Petascale Earthquake Modeling'' (ACI-1148493); ``Community Computational Platforms for Developing Three-Dimensional Models of Earth Structure'' (EAR-1226343); and ``Community Computational Platforms for Developing Three-Dimensional Models of Earth Structure, Phase II'' (EAR-1349180).  This research is also part of the Blue Waters sustained-petascale computing project, which is supported by NSF (awards OCI-0725070 and ACI-1238993) and the state of Illinois. Blue Waters is a joint effort of the University of Illinois at Urbana-Champaign and its National Center for Supercomputing Applications (NCSA). Computational support was possible through PRAC allocations supported by NSF awards: ``Petascale Research in Earthquake System Science on Blue Waters (PressOnBlueWaters)'' (ACI-0832698); and ``Extending the Spatiotemporal Scales of Physics-Based Seismic Hazard Analysis'' (ACI-1440085). This research used resources of the Argonne Leadership Computing Facility at Argonne National Laboratory, which is supported by the Office of Science of the U.S. Department of Energy under contract DE-AC02-06CH11357. Additional computational support was provided by the Extreme Science and Engineering Discovery Environment (XSEDE) program, which is supported by NSF (ACI-1053575). Some of the computations described here were performed on Kraken at the U.S. National Institute for Computational Sciences (NICS) and at the Center for High Performance Computing of the University of Southern California. SCEC is funded by NSF Cooperative Agreement EAR-1033462 and U.S. Geological Survey Cooperative Agreement G12AC20038. The SCEC contribution number for this paper is 2067.
