
\begin{abstract}

 Seismic velocity models are key components of current research efforts dedicated to advancing our knowledge of the Earth's crustal structure and its influence on ground response during earthquakes, including regional deep geology and local effects produced by the geometry, spatial distribution, and material composition of sediments in basins and valleys. The plethora of regional seismic velocity models, each with its own custom interface to query the model, presents a significant challenge to the computational seismology community for successfully incorporating them into wave propagation simulations. This paper proposes a method, implemented via the Unified Community Velocity Model (UCVM) software framework, by which these disparate models can be integrated under a single abstract interface, regardless of the data format and map projection used by a particular velocity model. UCVM is a collection of software tools and application programming interfaces designed to provide standardized access to multiple seismic velocity models used in seismology and geophysics research. Here, we review the background that led to the development of UCVM and its various software components and features, including examples of recent applications that use UCVM tools or output datasets in geoscience research.

\end{abstract}

