
\begin{abstract}

This paper reviews the Unified Community Velocity Model (UCVM) software framework developed by the Southern California Earthquake Center. UCVM is a collection of software tools and application programming interfaces designed to provide standardized access to multiple seismic velocity models used in seismology and geophysics research. Seismic velocity models are key components of current research efforts dedicated to advancing our knowledge of the Earth's crustal structure and its influence on ground response during earthquakes, including regional deep geology and local effects produced by the geometry, spatial distribution, and material composition of sediments in basins and valleys. The UCVM software framework has been designed to facilitate a broad range of research activities involving the use of seismic velocity models. Its development has been particularly useful to, and driven by, deterministic physics-based earthquake ground-motion simulation and seismic hazard analysis applications. UCVM has been extensively used in modeling and simulation activities in southern California. Here, we review the background that led to the development of UCVM and its various software components and features, including examples of recent applications that use UCVM tools or output datasets in geoscience research.

\end{abstract}

