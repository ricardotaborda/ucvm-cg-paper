
\subsection{Evaluation of CVMs through Validation}

One key aspect in earthquake ground motion simulation is the validation of synthetic seismograms against data recorded during past earthquakes. This helps test both models and simulation methods. A particular event of relevance is the 29 July 2008 \eqmag{w} 5.4 Chino Hills earthquake in southern California. This moderate earthquake, the strongest in the Los Angeles area since the 1994 Northridge earthquake, caused no significant damage or fatalities, but provided an excellent opportunity for earthquake research because it was recorded by over 500 seismic monitoring stations in California and other neighboring states. 

Various recent studies have included simulations of the 2008 Chino Hills earthquake with models created using the UCVM mesh generation utilities \citep[e.g.,][]{Olsen_2010_SRL, Taborda_2013_BSSA, Taborda_2014_BSSA}. These simulations have helped test current modeling approaches to predict the ground motion of moderate magnitude earthquakes at low ($<1$~Hz), medium (1--4~Hz), and high ($>4$~Hz) frequencies using deterministic and hybrid methods. They have as well as provided a context to propose and test different validation algorithms (goodness-of-fit criteria). More important, in the context of using UCVM, these simulations have helped evaluate the various CVMs available for the southern California region, and how their differences impact simulation results used for engineering applications.

\citet{Taborda_2014_BSSA}, in particular, focuses on the validation of simulations of the Chino Hills earthquake using different velocity models, namely CVM-S (v4) and CVM-H (v11.9.1). The discrete representations of these models used as input for their simulations were created using UCVM meshing programs. The authors evaluate the differences between these two models and their effect on validation results through comparisons with recorded seismograms from over 300 stations scattered throughout a simulation-domain area of \adomain{180}{135}{km}. Figure \ref{fig:ch.validation} shows comparisons between the surface \vs{} from the two models, the simulation outcome for peak ground velocity, the results of a goodness-of-fit validation analysis, and the differences of attenuation relationships derived from the simulations with respect to the data attenuation and empirical relationships typically used in engineering applications. Here, we highlight that the discrete input models used in these simulations were created utilizing the etree MPI utilities of UCVM and built on Kraken (now decommissioned) at the National Institute for Computational Sciences and on NCSA's Blue Waters supercomputer systems. The etree databases ranged between 110 and 260 GB, and the resulting finite-element models comprised meshes ranging between 5 billion and 15 billion octants.


\begin{figure*}[ht!]
	\centering
	\includegraphics
		[width=\textwidth]
		{figures/pdf/ch-validation}
	\caption{Evaluation of validation results obtained for simulations of the 2008 Chino Hills earthquake using two different velocity models, CVM-S v4 (top) and CVM-H v11.9.1 (bottom). The evaluation includes comparisons of the two input models' \vs{} surface values (far-left), the simulation results for the surface peak ground velocity (center-left), the outcome of a goodness-of-fit validation analysis with respect to data where lighter colors indicate more accurate results (center-right), and the comparison of results with respect to empirical attenuation relationships used in engineering applications (far-right). After \citet{Taborda_2014_BSSA}.}
	\label{fig:ch.validation}
\end{figure*}



