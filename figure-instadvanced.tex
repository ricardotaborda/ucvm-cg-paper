
\begin{figure}[ht!]
\centering
\begin{lstlisting}[
	frame=single,
	basewidth={0.45em,0.4em},
	backgroundcolor=\color{mylistingbkgd},
	basicstyle=\ttfamily\footnotesize,breaklines=true,
	linewidth=0.98\columnwidth,xleftmargin=0.07\columnwidth,
	numbers=left,numberblanklines=true,numberstyle=\scriptsize\color{mylistingnclr}]
> module swap PrgEnv-pgi PrgEnv-gnu
> module load iobuf
  [...]
> UCVM_DIR=<user defined path>
> UCVM_VERSION=14.3.0
> ROOT_URL=http://hypocenter.usc.edu/research/ucvm
> LIBS_URL=$ROOT_URL/$UCVM_VERSION/libraries
> MODELS_URL=$ROOT_URL/$UCVM_VERSION/models
  [...]
> cd $UCVM_DIR
> wget $LIBS_URL/proj-4.8.0.tar.gz
> wget $LIBS_URL/euclid3-1.3.tar.gz
> wget $MODELS_URL/cvms4.tar.gz
  [...]
> tar xzvf proj-4.8.0.tar.gz
> cd proj-4.8.0
> ./configure --prefix=$UCVM_DIR/lib/proj-4 --with-jni=no
> make
> make install
  [...]
> cd $UCVM_DIR
> tar xvzf euclid3-1.3.tar.gz
> cd euclid3-1.3
> ./configure --prefix=$UCVM_DIR/lib/euclid3
> make; make install
  [...]
> cd $UCVM_DIR
> tar xzvf cvms4.tar.gz
> cd CVM-S
> ./configure --prefix=$UCVM_DIR/model/cvms4
> make
> make install
  [...]
> cd $UCVM_DIR
> tar xzvf ucvm-14.3.0.tar.gz
> cd UCVM
> ./configure \
  --prefix=$UCVM_DIR \
  --enable-iobuf \
  --enable-static \
  --with-etree-include-path=$UCVM_DIR/lib/euclid3/include \
  --with-etree-lib-path=$UCVM_DIR/lib/euclid3/lib \
  --with-proj4-include-path=$UCVM_DIR/lib/proj4/include \
  --with-proj4-lib-path=$UCVM_DIR/lib/proj4/lib \
  --enable-model-cvms \
  --with-cvms-lib-path=$UCVM_DIR/model/cvms4/lib \
  --with-cvms-model-path=$UCVM_DIR/model/cvms4/src \
  --with-cvms-include-path=$UCVM_DIR/model/cvms4/src
> make
> make install
> make check
  [...]
\end{lstlisting}
\caption{Advanced installation procedure showing the commands a user will follow when working on a Cray supercomputer system with default PGI compilers and Lustre filesystem. Note that UCVM requires GNU GCC compilers, version 4.3+. This particular set of instructions installs the model CVM-S (version 4) with the Proj and Etree libraries using static build and enabling IO-buffering. Shell commands are indicated by the symbol \textgreater. \textcolor{red}{This example needs to be checked.}}
\label{fig:instadvanced}
\end{figure}


